\chapter{\tl{Quantum Algorithms}}

\section{\tl{Probabilistic Versus Quantum Algorithms}}
Οι πιθανοτικοί αλγόριθμοι έχουν ευρεία χρήση στη θεωρητική πληροφορική καθώς χρησιμοποιούνται στη λύση πολλών διαφορετικών προβλημάτων. Σε αυτό το κεφάλαιο θα δούμε πως είναι δυνατό οι κβαντικοί αλγόριθμοι να αποτελέσουν μια γενίκευση των πιθανοτικών αλγορίθμων.

Θα ξεκινήσουμε με ένα απλό παράδειγμα ενός πιθανοτικού αλγόριθμου. Το παρακάτω σχήμα απεικονίζει τα πρώτα δύο βήματα ενός τέτοιου υπολογισμού σε έναν καταχωρητή ο οποίος μπορεί να βρίσκεται σε μια από τις τέσσερις καταστάσεις. Αρχικοποιούμε τον καταχωρητή στο $0$. Μετά το πρώτο βήμα, ο καταχωρητής είναι στην κατάσταση $j$ με πιθανότητα $p_{0,j}$. Υποθέτωντας ότι θέλουμε να βρούμε τη συνολική πιθανότητα ότι ο υπολογισμός τερματίζει στην κατάσταση $3$ μετά το δεύτερο βήμα. Αυτός ο υπολογισμός γίνεται σε δύο στάδια. Αρχικά υπολογίζουμε την πιθαντότητα που σχετίζεται με τα διαφορετικά υπολογιστικά μονοπάτια τα οποία τερματίζουν στην κατάσταση $3$ και στη συνέχεια προσθέτωντας τις πιθανότητες όλων αυτών των διαφορετικών μονοπατιών. Μπορούμε να μεταβούμε από την κατάσταση $0$ στην κατάσταση $j$ και στη συνέχεια από την κατάσταση $j$ στην κατάσταση $3$ με ένα από τα τέσσερα $j \epsilon {0, 1, 2, 3}$. Όπως είναι ήδη γνωστό η πιθανότητα που σχετίζεται με καθένα από τα μονοπάτια βρίσκεται εάν πολλαπλασιάσουμε την πιθανότητα $p_{0,j}$ της μετάβασης από την $0$ στην $j$ με την πιθανότητα $q_{j,3}$ της μετάβασης από τη $j$ στην $3$. Η συνολική πιθανότητα να τερματίσουμε στην $3$ δίνεται αν προσθέσουμε τις τέσσερις πιθανότητες που προκύπτουν.
Όμως οι πιθανότητες αυτές που προκύπτουν είναι τα τετράγωνα των κβαντικών πιθανοτικών πλατών. Δηλαδή $p_{0,j} = \vert \alpha_{0,j} \vert ^2$ και $q_{j,k} = \vert \beta_{j,k} \vert^2$.
Άρα το τελικό αποτέλεσμα που προκύπτει είναι~\cite{Kaye:2007:IQC:1206629}:

\begin{align*}
  prob = \sum_{j} p_{0,j}q_{j,3}
\end{align*}

Όπως μπορούμε να δούμε, οι κλασσικοί πιθανοτικοί αλγόριθμοι μπορούν να προσομοιαστούν από τους κβανιτκούς αλγορίθμους. Είναι εφικτή όμως η αντίστροφη διαδικασία; Η αντικατάσταση μιας κβαντικής πύλης με μια πιθανοτική κλασσική πύλη μπορεί να δώσει τελείως διαφορετικά αποτελέσματα, άρα δεν είναι μια εφικτή λύση. Ωστόσο σε κάποιες συγκεκριμένες περιπτώσεις όπως τα κβαντικά κυκλώματα που χρησιμοποιούν μόνο πύλες $CNOT, H, X, Y, Z, T$, μπορούν να προσομοιαστούν σε έναν κλασσικό υπολογιστή. Μπορούμε να συμπεράνουμε ότι είναι πολύ πιθανό οι κβαντικοί αλγόριθμοι να επιλύουν πολύ πιο γρήγορα τα όποια προβλημάτα σε σχέση με τους κλασσικούς πιθανοτικούς αλγορίθμους.

\begin{tikzpicture}[auto,node distance=1.5cm]
  \node[entity] (node1) {Fancy Node 1}
    [grow=up,sibling distance=3cm]
    child {node[attribute] {Attribute 1}}
   child {node[attribute] {Attribute 2}}
   child[grow=left,level distance=3cm] {node[attribute] {Attribute 3}};
 % Now place a relation (ID=rel1)
 \node[relationship] (rel1) [below right = of node1] {Relation 1};
 % Now the 2nd entity (ID=rel2)
 \node[entity] (node2) [above right = of rel1]	{Fancy Node 2};
 % Draw an edge between rel1 and node1; rel1 and node2
 \path (rel1) edge node {1-\(m\)} (node1)
   edge	 node {\(n\)-\(m\)}	(node2);
\end{tikzpicture}

\section{\tl{Quantum Parallelism}}

\section{\tl{Deutsch's Algorithm}}

\section{\tl{Deutsch-Jozsa Algorithm}}

\section{\tl{Simon's Algorithm}}

\section{\tl{Grover's Algorithm}}

\section{\tl{Quantum Search Algorithms}}
