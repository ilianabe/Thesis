\chapter{\selectlanguage{english}The quantum Fourier Transformation}
\section{\selectlanguage{english}FFT}
Οι πράξεις μεταξύ πολυωνύμων μεγάλου μεγέθους χρησιμοποιούνται ευρέως. Η ολοκλήρωση τους "χειροκίνητα" είναι μια χρονοβόρα και δύσκολη διαδικασία. Η επίλυση σε αυτό έρχεται με τη χρήση του Γρήγορου Μετασχηματισμού~\cite{fft} $Fourier\  (FFT)$.\\
Ο $FFT$, βασίζεται σε κάποιες βασικές αρχές και ιδιότητες~\cite{DBLP:books/daglib/0017733}. Η κυριότερη είναι η ιδιότητα των πολυωνύμων να χαρακτηρίζονται με δύο διαφορετικές μορφές.\\
Το πολυώνυμο $A(x)$ μπορεί να χαρακτηριστεί με τους εξής δύο τρόπους:
\begin{itemize}
    \item Μέσω των συντελεστών του $α_0, α_1,...,α_d$
    \item Μέσω των τιμών του $A(x_0), A(x_1),...,A(x_d)$
\end{itemize}

Πολύ συνοπτικά, ο $FFT$ είναι ένας αλγόριθμος διαίρει-και-βασίλευε, που βασίζεται στις ιδιότητες των μιγαδικών ριζών της μονάδας. Χρησιμοποιεί τον διακριτό μετασχηματισμό $Fourier$ $(DFT)$ και τον αντίστροφο $DFT$ για να μετατρέπει τις δύο μορφές αναπαράστασης του πολυωνύμου. Από την αναπαράσταση του πολυωνύμου βαθμού $d$ μέσω συντελεστών, προκύπτει ένα διάνυσμα με τους συντελεστές $\alpha = (\alpha_0, \alpha_1,...,\alpha_d)$. Η απεικόνιση ενός πολυωνύμου βαθμού $d$ σε μορφή τιμών μας δίνει ένα πακέτο από $d$ ζευγάρια της μορφής {$(x_0,y_0),(x_1,y_1),...,(x_d,y_d)$} τέτοια ώστε κάθε $x_k$ να είναι διακριτοί αριθμοί και κάθε $y_k = A(x_k)$ για κάθε $k = 0,1,...,d$.

Η κεντρική ιδέα πάνω στην οποία βασίστηκε ο $FFT$, είναι η δυνατότητα κάθε πολυωνύμου βαθμού $d$, να μπορεί να χαρακτηριστεί μοναδικά από την τιμή του σε κάθε $d+1$ σημείο του. Έστω λοιπόν ότι έχουμε ένα γινόμενο $C(x)$, δύο πολυωνύμων $A(x)$ και $B(x)$ βαθμού $d$ το καθένα. Αυτό σημαίνει ότι το $C(x)$, θα έχει βαθμό $2d$. Με βάση λοιπόν την παραπάνω ιδιότητα, το $C(x)$, θα μπορεί να χαρακτηριστεί μοναδικά σε κάθε $2d+1$ σημείο του. Εάν λοιπόν το $x$ είναι γνωστό, έστω $z$, ο πολλαπλασιασμός είναι μια διαδικασία γραμμικού χρόνου, καθώς το αποτέλεσμα θα προκύπτει εάν κάνουμε $A(z)$ φορές το $B(z)$. Ο πολλαπλασιασμός δύο πολυωνύμων $A(x)$ και $B(x)$ βαθμού $d$ το καθένα, θέλει χρόνο $\theta(d^2)$, καθώς κάθε συντελεστής στο διάνυσμα $\alpha$ πρἐπει να πολλαπλασιαστεί με κἀθε συντελεστή στο διάνυσμα $\beta$. Το διάνυσμα συντελεστών του γινομένου $C(x)$, $c = (c_0,c_1,...,c_{2d}$ ονομάζεται $convolution$ των διανυσμάτων $\alpha$ και $\beta$ και γράφεται και ως $c = \alpha \otimes \beta$.

Όμως εδώ έχουμε να αντιμετωπίσουμε το πρόβλημα, ότι και τα πολυώνυμα εισόδου αλλά και το γινόμενο θέλουμε να είναι σε μορφή συντελεστών. Επομένως πρέπει να γίνουν δύο μετατροπές. Μια μετατροπή από την μορφή των συντελεστών στην μορφή των τιμών $(evaluation)$ τα επιλεγμένα σημεία στα οποία θα κάνουμε τον πολλαπλασιασμός και στη συνέχεια μετατρέπουμε ξανά το τελικό πλέον αποτέλεσμα σε μορφή συντελεστών $(interpolation)$. Η διαδικασία $evaluation$ στο σημείο $x_0$, αποτελείται από τον υπολογισμό της τιμής $A(x_0)$.  Χρησιμοποιώντας το σχήμα $Horner$ βρίσκουμε την τιμή σε χρόνο $\theta(n)$, ως εξής:

\begin{align*}
    A(x_0) = \alpha_0 + x_0(\alpha_1 + x_0(\alpha_2 +...+x_0(\alpha_d)...))
\end{align*}

Η αντίστροφη διαδικασία, δηλαδή η $interpolation$ καθορίζει την τιμή του συντελεστή του πολυωνύμου, από την τιμή που αναπαρίσταται. Για κάθε "πακέτο" $(x_0,y_0),(x_1,y_1),...,(x_d,y_d)$, όπου κάθε $x_k$ είναι διακριτός αριθμός, υπάρχει ένα μοναδικό πολυώνυμο $A(x)$ βαθμού $d$ για το οποίο ισχύει ότι $y_k = A(x_k)$, για κάθε $k=0,1,...,d$. Χρησιμοποιώντας τη μέθοδο $Lagrange$, μπορούμε να κάνουμε την $interpolation$ σε χρόνο $\theta(d^2)$.

Το αμέσως επόμενο ερώτημα που προκύπτει είναι το πως ακριβώς θα επιλεγούν τα σημεία με βαθμό $<n-1$ του $A(x)$, στα οποία θα εφαρμόσουμε τους μετασχηματισμούς. Για να μειώσουμε τις πράξεις κατά το μέγιστο δυνατό, επιλέγουμε ζευγάρια θετικών-αρνητικών αριθμών, $\pm x_i,..., \pm x_{\frac{n}{2}-1}$. Με πιο απλά λόγια, χωρίζουμε το πολυώνυμο $A(x)$, με αυτόν τον τρόπο:

\begin{align*}
    A(x) = A_e(x^2) + xA_o(x^2)
\end{align*}
, όπου $Α_e$ ο συντελεστής των άρτιων δυνάμεων και $A_o$ ο συνεντελεστής των περιττών δυνάμεων. Ο υπολογισμός ενός ζευγαριού $\pm x_i$ γίνεται ακόμα πιο έυκολος καθώς ισχύει η εξής ιδιότητα:
\begin{align*}
    A(x_i) = A_e(x_i^2) + x_iA_o(x_i^2) και
    A(x_i) = A_e(-x_i^2)-x_iA_o(x_i^2)
\end{align*}


Κάνοντας $evaluation$ του $Α(x)$ σε $n$ θετικά αρνητικά σημεία $\pm x_0,...,x_{\frac{n}{2}-1}$ μειώνει την διαδικασία στα $Α_e(x)$ και $A_o(x)$ σε $\frac{n}{2}$ σημεία. Έτσι λοιπόν το αρχικό πρόβλημα μήκους $n$ μειώνεται σε δύο υποπροβλήματα μήκους $\frac{n}{2}$ το καθένα και σε κάποιες σχετικά απλές πράξεις. Ο συνολικός χρόνος εκτέλεσης είναι $T(n) = 2T(\frac{n}{2}) + O(n)$, το οποίο ανάγεται σε $O(nlogn)$, που είναι ένας πολύ ικανοποιητικός χρόνος.

Η τεχνική αυτή με την επιλογή θετικών-αρνητικών ζευγαριών εφαρμόζεται στο πρώτο επίπεδο. Για να προχωρήσουμε στο επόμενο επίπεδο θέλουμε $\frac{n}{2}$ σημεία $x_0^2,...,x_{\frac{n}{2}-1 ^2} $  να είναι τα ίδια θετικά-αρνητικά ζευγάρια. Δεδομένου όμως ότι τα σημεία αυτά είναι υψωμένα στο τετράγωνο, είναι αδύνατο χωρίς την χρήση μιγαδικών αριθμών. Στο τελευταίο επίπεδο της αναδρομής θα έχουμε μόνο ένα σημέιο, το $\pm 1$. Στο ακριβώς προηγούμενο επίπεδο της αναδρομής θα έχουμε τις ρίζες του $\pm 1$, δηλαδή $\pm i$. Συνεχίζουμε έτσι σε κάθε επίπεδο μέχρι που καταλήγουμε στην $n-οστή$ ρίζα του συνόλου, η οποία είναι οι μιγαδικές ρίζες της εξίσωσης $z_n = 1$.

Μέχρι αυτό το σημείο έχουμε δει πως γίνεται η μετατροπή σε τιμές και ο πολλαπλασιασμός αυτών. Ο $FFT$, μετατρέπει από της μορφή συντελεστών στη μορφή τιμών σε  $O(n logn)$ όταν τα σημεία $x_i$ είναι οι μιγαδικές $n$-οστές ρίζες του 1 $(1, \omega, \omega^2,...,\omega^{n-1})$.
Σχηματικά ισχύει το εξής:

\begin{align*}
    \langle values\rangle = FFT(\langle coefficients \rangle, \omega).
\end{align*}
Ωστόσο δεν μπορούμε να αγνοήσουμε τους συντελεστές, καθώς σε αυτή τη μορφή μας δίνονται όλα τα δεδομένα μας. Η τελευταία αυτή μετατροπή γίνεται με τη διαδικασία $intepolation$.
\begin{align*}
    \langle coefficients \rangle = \frac{1}{n}FFT(\langle values \rangle, \omega ^{-1}).
\end{align*}

Για να αποκτήσουμε μια καλύτερη εικόνα της $interpolation$, πρέπει να δούμε λίγο πιο αναλυτικά τη σχέση ανάμεσα στις δυο διαφορετικές απεικονίσεις του $A(x)$. Και οι δύο μορφές αποτελούν διανύσματα $n$ αριθμών και η κάθε απεικόνιση είναι ο γραμμικός μετασχηματισμός της άλλης.\\

$\begin{bmatrix}
A(x_{0})\\
A(x_{1})\\
\vdots\\
A(x_{n-1})
\end{bmatrix}$
=
$\begin{bmatrix}
1 & x_{0} & x_{0}^2 & \dots & x_{0}^{n-1}\\
1 & x_{1} & x_{1}^2 & \dots & x_{1}^{n-1}\\
\ & \ & \vdots & \ & \ \\
1 & x_{n-1} & x_{n-1}^2 & \dots & x_{n-1}^{n-1}\\
\end{bmatrix}$
$\cdot$
$\begin{bmatrix}
\alpha_{0}\\
\alpha{1}\\
\vdots\\
\alpha_{n-1}
\end{bmatrix}$\\

Ο μεσαίος πίνακας ονομάζεται $M$ και έχει κάποιες συγκεκριμένες ιδιότητες. Εάν τα $x_0,...x_{n-1}$ είναι διακριτοί αριθμοί, τότε ο $M$ είναι αντιστρέψιμος. Η ύπαρξη του $M^{-1}$, μας δίνει τη δυνατότητα να αντιστρέψουμε την εξίσωση μήτρας και να εκφράσουμε την μορφή των συντελεστών σε μορφή τιμών. Με λίγα λόγια όταν κάνουμε $evaluating$ πολλαπλασιάζουμε με τον $Μ$ και όταν κάνουμε $interpolation$, πολλαπλασιάζουμε με τον $M^{-1}$.

Ας προσπαθήσουμε να εξηγήσουμε λίγο μαθηματικά τον $FFT$. Με όρους γραμμικής άλγεβρας, ο γρήγορος μετασχηματισμός $Fourier$, έναν αυθαίρετο $vector$, διαστάσεως $n$ (ο οποίος αποτελείται από τους συντελεστές του πολυωνύμου) με έναν πίνακα $n \times n$ της παρακάτω μορφής:\\
$M_n(\omega)$ =
$\begin{bmatrix}
1 & 1 & 1 & \dots & 1\\
1 & \omega & \omega^2 & \dots & \omega^{n-1}\\
1 & \omega^2 & \omega^4 & \dots & \omega^{2(n-1)}\\
\ & \ & \vdots & \ & \ \\
1 & \omega^j & \omega^2j & \dots & \omega^{(n-1)j}\\
\ & \ & \vdots & \ & \ \\
1 & \omega^{(n-1)} & \omega^{2(n-1)} & \dots & \omega^{(n-1)(n-1)}\\
\end{bmatrix}$
, όπου η πρώτη σειρά\\ προκύπτει για $\omega^0 = 1$, η δεύτερη για $\omega$, η τρίτη για $\omega^2$ μέχρι την τελευταία που προκύπτει για $\omega^(n-1)$. Το $\omega$ είναι η $n$-οστή μιγαδική ρίζα του 1 και το $n$ είναι μια δύναμη του 2. Βλέπουμε ότι ο συγκεκριμένος πίνακας είναι πολύ απλό να περιγραφεί καθώς σε κάθε $(j,k)$ θέση βρίσκεται το $\omega^{jk}$. Αυτό που μπορούμε να παρατηρήσουμε σε αυτό το στάδιο είναι ότι οι στήλες του $M$ είναι ορθογώνιες μεταξύ τους. Αν πάρουμε το αποτέλεσμα από δύο τυχαίες στήλες του $M$, έστω $j$ και $k$ τότε προκύπτει το εξής:
\begin{align*}
    1 + \omega^{j-k} + \omega^{2(j-k)} +...+ \omega^{(n-1)(j-k)}
\end{align*}
, το οποίο είναι γεωμετρική σειρά με πρώτο όρο το $1$ και τελευταίο το $\omega^{(n-1)(j-k)}$. Έπομένως μετατρέπεται στο $(1-\omega^{n(j-k)})/(1-\omega^{(j-k)})$, το οποίο είναι $0$ για κάθε τιμή εκτός από $j=k$, όπου σε αυτή την περίπτωση όλοι οι όροι είναι $1$ και το τελικό σύνολο $n$.
Οι στήλες αυτές θα μπορούσαν να θεωρηθούν ως η βάση ενός εναλλακτικού συστήματος συντεταγμένων, το οποίο συχνά αποκαλείται βάση $Fourier$. Ο πολλαπλασιασμός ενός διανύσματος με τον $Μ$, οδηγεί στην περιστροφή του κλασσικού συστήματος συντεταγμένων στο σύστημα βάσης $Fourier$. Ο αντίστροφος $M$, προκαλεί την αντίστροφη περιστροφή. Με λίγα λόγια ισχύει ότι: $M_n(\omega^{-1}) = \frac{1}{n}M_n(\omega^{-1})$. Όμως το $\omega^{-1}$ είναι και η $n$-οστή ρίζα της μονάδας και έτσι κάνοντας $interpolation$ επί της ουσίας κάνουμε $FFT$, μόνο που αντί για $\omega$ έχουμε $\omega^{-1}$. Κοιτάζοντας λοιπόν συνολικά μέχρι εδώ βλέπουμε ότι και από γεωμετρικής άποψης, ο πολλαπλασιασμός μεγάλων πολυωνύμων είναι αρκετά πιο εύκολος στην βάση $Fourier$, από ότι στην κλασσική βάση. Αρχικά περιστρέφουμε τα διανύσματα σε βάση $Fourier$ $(evaluation)$, στη συνέχεια κάνουμε την πράξη που θέλουμε (στην προκειμένη περίπτωση πολλαπλασιασμό) και τέλος περιστρέφουμε τα διανύσματα ξανά αντίστροφα $(interpolation)$. Τα αρχικά διανύσματα είναι η απεικόνιση σε μορφή συντελεστών, όταν περιστρέφονται μετατρέπονται σε μορφή τιμών και μετά την αντίστροφη περιστροφή επανέρχονται σε μορφή συντελεστών. Η γρήγπρη εναλλαγή μεταξύ των δύο αυτών καταστάσεων είναι ο γρήγορος μετασχηματισμός $Fourier$.

Αυτό είναι το συνολικό υπόβαθρο του $FFT$. Ωστόσο το πιο ενδιαφέρον κομμάτι του είναι η υπορουτίνα που κάνει αυτή την εναλλαγή που είδαμε πιο πάνω. Ο $FFT$ παίρνει σαν είσοδο ένα διάνυσμα $\alpha = (\alpha_0,...,\alpha_{n-1})$ και έναν μιγαδικό αριθμό $\omega$, του οποίου οι δυνάμεις αποτελούν τις μιγαδικές $n$οστές ρίζες της μονάδας. Πολλαπλασιάζει το διάνυσμα με τον πίνακα $M_n(\omega)$ διάστασης $n \times n$, ο οποίος έχει σαν είσοδο σε κάθε $j,k$ σημείο του το αντίστοιχο $\omega^{jk}$. Ο διαχωρισμός που αναλύσαμε παραπάνω σε ζεύγη θετικών αρνητικών είναι πολύ βοηθητικός σε αυτό ακριβώς το σημείο καθώς οι στήλες του $M_n$ χωρίζονται σε αρνητικούς και θετικούς. Στο επόμενο βήμα απλοποιούμε τα στοιχεία στο κάτω μισό του πίνακα χρησιμοποιώντας τα $\omega^{n/2} = -1$ και $\omega^n = 1$. Το πάνω αριστερά κομμάτι του πίνακα όπως και το κάτω αριστερά με διάσταση $n/2 \times n/2$ είναι το $M_{n/2}(\omega^2)$.
Επίσης ο πάνω δεξιά και ο κάτω δεξιά υποπίνακας είναι σχεδόν ίδιοι με τους προηγούμενους, μόνο που οι $j$οστές σειρές τους είναι πολλαπλασιασμένες με το $\omega^j$ και $-\omega^j$, αντίστοιχα. Έτσι λοιπόν το τελικό αποτέλεσμα είναι ακι το ζητούμενο διάνυσμα μας.

Συνοπτικά ο $FFT$ έχει ώς εξής:


\begin{algorithm}
\selectlanguage{english}

\caption{Fast Fourier Transform}
\begin{algorithmic}[1]
\State $function FFT(\alpha, \omega)$
\State $Input:\ An \ array \ \alpha = (\alpha_0,...\alpha_{n-1}), \ for\ n \ a \ power \ of \ 2$
\State $ \qquad \qquad A \ primitive \ nth\ root\ of\ unity,\ \omega$
\State $Output: \ M_n(\omega) \alpha$\\


\If {$\omega = 1$} \Return $\alpha$
\EndIf

\State $(s_0,s_1,...,s_{\frac{n}{2}-1} = FFT ((\alpha_0,\alpha_2,...,\alpha_{n-2}, \omega^2$
\For {$j= 0 \rightarrow \frac{n}{2}-1$}
        \State $r_j = s_j + \omega^js'_j$
        \State $r_{j+\frac{n}{2}} = s_j - \omega^js'_j$
\EndFor
\Return $(r_0,r_1,...,r_{n-1}$


\end{algorithmic}
\end{algorithm}


\subsection{Υλοποίηση $FFT$}

Παραπάνω πήραμε μια γενική ιδέα του αλγορίθμου. Όπως είδαμε ο $FFT$ μπορεί να χωριστεί σε δύο βασικά βήματα. Αρχικά θέλουμε να φέρουμε τα δύο πολυώνυμα $A(x)$ και $B(x)$ στην κατάλληλη μορφή συντελεστών ($coefficients$) εφαρμόζοντας τον $FFT$, στη συνέχεια να τα πολλαπλασιάσουμε ανά στοιχείο και τέλος να μετατρέψουμε το τελικό διάνυσμα που έχουμε βρει στη σωστή μορφή με χρήση του $inverse FFT (iFFT)$, όπου θα αποτελεί τους συντελεστές του πολυωνύμου του γινομένου $C(x)$. Το πρώτο βήμα ονομάζεται $evaluation$ και το δεύτερο $intepolation$. Για το $evaluation$ θα χρειαστούμε ένα πολύ βασικό στοιχείο, τον αριθμό των σημείων τα οποία θα κάνουμε $evaluate$. Ο αριθμός αυτός προκύπτει από τις $n+1$ μιγαδικές ρίζες της μονάδας, όπου $n$ ο βαθμός του πολυωνύμου του γινομένου $C(x)$. Οι μιγαδικές αυτές ρίζες προκύπτουν ως εξής: $\omega^n = 1$, $e^{\frac{2i\pi k}{n}}, k=0,..,n-1$, όπου ξέρουμε ότι: $e^iu=cos(u)+isin(u)$ με το $u$ να είναι μια γωνία στο καρτεσιανό σύστημα συντεταγμένων.

\begin{exmp}
Ας δούμε και ένα απλό παράδειγμα βήμα βήμα για το πως δουλεύει ο αλγόριθμος.

Έστω ότι έχουμε τα δύο πολυώνυμα $A(x)=4x^2+4$ και $B(x)=3x+2$. Ο βαθμός του γινομένου $C(x)$ θα είναι 3, άρα $n=4$. Οι 4 μιγαδικές ρίζες της μονάδας είναι $\omega^4=1\rightarrow \omega=1,i,-1,-i$.
\\
Για το πολυώνυμο Α: \\
Οι συντελεστές είναι (4,0,4,0), άρα για ο μετασχηματισμός για τις ζυγές θέσεις θα είναι $Even_A:(4,4)\mapsto(8,0)$ και για τις μονές θέσεις θα είναι: $Odd_A:(0,0)\mapsto(0,0)$.\\
Άρα ο μετασχηματισμός $Fourier$ του πολυωνύμου $A$ θα είναι: (8, 0, 8, 0) το οποίο προκύπτει από τον υπολογισμό του συντελεστή περιστροφής, $X_{0,2}=E_0\pm O_0$ και $X_{1,3} = E_1 \mp \imath O_1$ (\href{https://www.dsprelated.com/showarticle/107.php}{$Twiddle Factor Calculation$}).\\

Κάνουμε το ίδιο και στο δεύτερο πολυώνυμο, όπου το διάνυσμα συντελεστών είναι: (2,3,0,0). Ο μετασχηματισμός στις ζυγές και στις μονές θέσεις θα είναι: $Even_B:(2,0)\mapsto(2,2)$ και $Odd_B:(3,0)\mapsto(3,3)$.\\
Άρα ο μετασχηματισμός $Fourier$ του πολυωνύμου $B$ θα είναι: ($5,2-3\imath ,-1, 2+3\imath$).\\
Βρίσκουμε το μετασχηματισμό $Fourier$ του πολυωνύμου $C(x)$, πολλαπλασιάζοντας τους συντελεστές των $A$ και $B$. Οι συντελεστές του $C$ είναι: (40, 0, -8, 0). Όπως είπαμε και παραπάνω, πρέπει να εφαρμόσουμε τον αντίστροφο μετασχηματισμό $(iFFT)$, για να "επαναφέρουμε" τους συντελεστές στην αρχική μορφή. Μια από τις πιο εντυπωσιακές ιδιότητες του $FFT$ είναι ότι μπορεί να χρησιμοποιηθεί και προς τις δύο κατευθύνσεις με μερικές ελάχιστες διαφοροποιήσεις. Αυτή τη στιγμή έχουμε το αποτέλεσμα του $interpolation$ και μπορούμε να εφαρμόσουμε τον $FFT$ σε αυτό ακριβώς το διάνυσμα. Οι μόνες δύο διαφορές είναι οι εξής:
\begin{enumerate}
  \item Αντί για το $\omega$ που χρησιμοποιήσαμε για να πάρουμε το $interpolated$ διάνυσμα, στην αντίστροφη διαδικασία θα χρησιμοποιήσουμε στη θέση του το $\frac{1}{\omega}$
  \item Το αποτέλεσμα θα είναι $n$ φορές το διάνυσμα των συντελεστών, οπότε θα πρέπει να κάνουμε μια διαίρεση με το $n$.
\end{enumerate}

% Άρα για το $C$:\\
%
% $Even_C: (40,-8) \mapsto (32,48)$ και $Odd_C: (0,0)\mapsto (0,0)$. Οι αρχικές τιμές των συντελεστών θα είναι $\frac{1}{4} * (32, 48, 32, 48) = (8, 12, 8, 12)$. Άρα το γινόμενο των δύο πολυωνύμων $A$ και $B$ είναι: $\textbf{C(x) = 12x^3 + 8x^2 + 12x + 8}$
\end{exmp}
Ο ψευδοκώδικας για τον αλγόριθμο του $FFT$ είναι:

\begin{algorithm}
  \selectlanguage{english}
  \caption{Fast Fourier Transform}
  \begin{algorithmic}
    \State FFT(a):
    \State   n = length(a);\ //a\ is\ the\ coefficients'\ vector
    \If {n = 1}
      \Return a
    \EndIf

    \State $w = e^{(2*pi*i/n)};$\ //the\ nth\ root\ of\ the\ universe\
    \State o = 1;
    \State  //Even\ and\ Odd\ coefficients\
    \State $Even_A = [A0, ..., An-2];$
    \State $Odd_A = [A1, ..., An-1];$
    \State   $x0 = FFT(Even_A);$
    \State   $x1 = FFT(Odd_A);$
    \For {$k = 0 \rightarrow n/2 -1$}
      \State $x[k] = x0[k] + o*x1[k];$
      \State $x[k+(n/2)] = x0[k] - o*x1[k];$
      \State o = o*w;
    \EndFor
    \Return x
  \end{algorithmic}
\end{algorithm}







Για την υλοποίηση του θα χρειαστούμε και κάποια πράγματα από τους μιγαδικούς αριθμούς.
Ένας μιγαδικός αριθμός, μπορεί να γραφτεί με δύιο τρόπους: $z = x + yi$ και $z = re^{\theta*i}$, όπου $r=\sqrt{x^2+y^2}$ και $\theta = arctan(y/x)$. Ένα πολύ σημαντικό βήμα στο αλγόριθμο είναι η επιλογή των $n$ σημείων τα οποία θα υπολογίσουμε. Πρέπει $n \geq 2d+1$, όπου $d$ ο βαθμός του πολυωνύμου. Με βάση το $n$, υπολογίζουμε τις $n-$οστές ρίζες της μονάδας, όπου είναι οι μιγαδικοί αριθμοί $1, \omega, \omega^2, ... , \omega^{n-1}$, όπου $\omega = e^{2*pi*i/n}$, όπως είδαμε και παραπάνω.
\\
Τα βασικά "στοιχεία" για την υλοποίηση του $FFT$, είναι μια

% \section*{$Quantum Fourier Transform$}
\section{Κβαντική Διεμπλοκή}
Η κβαντική διεμπλοκή έχει τις ρίζες της σε ένα άρθρο των \begin{otherlanguage}{english} A. Einstein, B. Podolsky \end{otherlanguage} και \begin{otherlanguage}{english}N. Rosen \end{otherlanguage}~\cite{Bell:111654}. Ο σκοπός του άρθρου ήταν να αποδείξουν ότι η κβαντική μηχανική δεν είναι μια πλήρης φυσική θεωρία, αλλά ότι από την κβαντική περιγραφή της φύσης λείπουν κάποιες παράμετροι. Αργότερα οι παράμετροι αυτές ονομάστηκαν "κρυμμένες μεταβλητές". Σαν μοντέλο για την απόδειξη τους, χρησιμοποίησαν ένα θεωρητικό πείραμα στο οποίο δύο κβαντικλα συστήματα, αφού αλληλεπιδράσουν μεταξύ τους απομακρύνονται το ένα από το άλλο. Τα δύο αυτά κβαντικά συστήματα παραμένουν διασυνδεδεμένα το ένα με το άλλο με έναν άγνωστο μη κλασσικό τρόπο. Αυτό έχει σαν αποτέλεσμα η μέτρηση μιας φυσικής ποσότητας του ενός, καθορίζει το αποτέλεσμα της μέτρησης της ίδιας φυσικής ποσότητας του άλλου. Το θεωρητικό αυτό πείραμα ονομάστηκε "παράδοξο $EPR$", από τα αρχικά των τριών ερευνητών.
Η κβαντική διεμπλοκή είναι ίσως η πιο αινιγματική πλευρά της κβαντικής μηχανικής και δεν έχει
κλασικό ανάλογο. Κάθε χρόνο πολλές δεκάδες άρθρα δημοσιεύονται σε επιστημονικά περιοδικά και
περιγράφουν επιστημονικές εργασίες που έχουν ως στόχο την κατανόηση, το χειρισμό και τον υπολογισμό της κβαντικής διεμπλοκής.
Για τους κβαντικούς υπολογιστές η κβαντική διεμπλοκή είναι ένας φυσικός πόρος, όπως η ενέργεια,
τον οποίο μπορούμε να χρησιμοποιήσουμε για να εκτελέσουμε κβαντικούς υπολογισμούς και να αναπτύξουμε
κβαντικούς αλγορίθμους~\cite{Nielsen:2011:QCQ:1972505}. Αυτό που έχει δηλαδή σημασία, δεν είναι να κατανοήσουμε τη φύση της κβαντικής διεμπλοκής (πράγμα που είναι ίσως αδύνατο), αλλά να μάθουμε να την παράγουμε και να τη χρησιμοποιούμε.

\selectlanguage{english}
\begin{remark}
  \selectlanguage{greek} Δύο κβαντικά συστήματα βρίσκονται σε κβαντική διεμπλοκή, όταν η κατάσταση τους δεν μπορεί να γραφεί ως τανυστικό γινόμενο των βασικών τους καταστάσεων.
\end{remark}

\selectlanguage{greek}
Για να κατανοήσουμε καλύτερα τον παραπάνω ορισμό, θεωρούμε ότι τα δύο κβαντικά συστήματα είναι αυτά τα δύο $qubits$, $\vert q_{s0} \rangle, \vert q_{s1} \rangle$, που βρίσκονται στην κατάσταση $\vert q_s \rangle$, η οποία είναι η εξής:

\begin{align*}
  \vert q_s \rangle = \frac{1}{\sqrt{2}}(\vert 10 \rangle + \vert 11 \rangle)
\end{align*}

Όμως μπορούμε να γράψουμε την $\vert q_s \rangle$, μπορεί να γραφεί ως εξής:

\begin{align*}
  \vert q_s \rangle = \frac{1}{\sqrt{2}}(\vert 10 \rangle + \vert 11 \rangle) = \vert 1 \rangle \otimes [\frac{1}{\sqrt{2}}(\vert 0 \rangle + \vert 1 \rangle)]
\end{align*}

Αυτό σημαίνει ότι οι καταστάσεις των $\vert q_{s0} \rangle$ και $\vert q_{s1} \rangle$ είναι:

\begin{align*}
  \vert q_{s1} \rangle = \vert 1 \rangle, \vert q_{s0} = \frac{1}{\sqrt{2}}(\vert 0 \rangle + \vert 1 \rangle)
\end{align*}

και τελικά
\begin{align*}
  \vert q_s \rangle = \vert q_{s1} \rangle \otimes \vert q_{s0} \rangle
\end{align*}

Πρακτικά η παραπάνω σχέση μας λέει ότι η $\vert q_s \rangle$ μπορεί να γραφεί ως τανυστικό γινόμενο των καταστάσεων των δύο $qubits$, δηλαδή τα $qubits$ δε βρίσκονται σε κβαντική διεμπλοκή αλλά σε υπέρθεση καταστάσεων.

Ας θεωρήσουμε τώρα δύο άλλα $qubits$ το $\vert q_{e0} \rangle$ και $\vert q_{e1} \rangle$. Αυτά τα δύο νέα $qubits$, βρίσκονται στην κατάσταση $\vert q_e \rangle$, η οποία είναι:
\begin{align*}
  \vert q_e \rangle = \frac{1}{\sqrt{2}}(\vert 00 \rangle + \vert 11 \rangle)
\end{align*}

Σε αυτή την περίπτωση όμως, η $\vert q_e \rangle$ δεν μπορεί να γραφεί σαν τανυστικό γινόμεντο των καταστάσεων των δυο $qubits$, άρα αυτά βρίσκονται σε κβαντική διεμπλοκή.

\subsection{Κβαντική Διεμπλοκή και Υπέρθεση}
Είδαμε λοιπόν ότι ανάμεσα στην κβαντική διεμπλοκή και την υπέρθεση υπάρχουν κάποιες βασικές διαφορές. Εάν μετρήσουμε την κατάσταση του $qubit \vert q_{s1} \rangle$ της κατάστασης $\vert q_s \rangle$, θα δούμε σίγουρα ότι βρίσκεται στην κατάσταση $\vert 1 \rangle$. Το $\vert q_{s0} \rangle$, μετά από αυτή τη μέτρηση, μπορεί να βρίσκεται είτε στην κατάσταση $\vert 0 \rangle$ είτε στην κατάσταση $\vert 1 \rangle$ με πιθανότητα 0.5, για την κάθε περίπτωση. Που σημαίνει ότι η μέτρηση του ενός δεν καθορίζει την κατάσταση του άλλου. Αυτό έρχεται σε αντίθεση με τα αποτελέσματα του θεωρητικού πειράματος που οδήγησε στο "παράδοξο $EPR$", άρα δεν μιλάμε για κβαντική διεμπλοκή.

Αν όμως μετρήσουμε την κατάσταση του $\vert q_{e1}$ της κατάστασης $\vert q_e \rangle$, θα βρούμε ότι βρίσκεται στην κατάσταση $\vert 0 \rangle$ με πιθανότητα 0.5 και στην κατάσταση $\vert 1 \rangle$ με πιθανότητα πάλι 0.5. Αν το βρούμε στην κατάσταση $\vert 0 \rangle$ και μετρήσουμε το $\vert q_{e0} \rangle$, θα το βρούμε σίγουρα στην κατάσταση $\vert 0 \rangle$. Αντίστοιχα αν το $\vert q_{e1} \rangle$, είναι στην κατάσταση $\vert 1 \rangle$, τότε και το $\vert q_{e0} \rangle$, θα είναι στην κατάσταση $\vert 1 \rangle$. Αυτό σημαίνει, ότι η μέτρηση του ενός $qubit$ καθορίζει το άλλο, άρα βρίσκονται σε κβαντική διεμπλοκή.

\section{\selectlanguage{english}Quantum Fourier Transformation}
Όπως έχουμε αναφέρει και παραπάνω η μεγαλύτερη δύναμη των κβαντικών υπολογιστών, είναι η δυνατότητα να επιτελέσουν πράξεις και να επιλύσουν προβλήματα που δεν είναι δυνατό με τους κλασσικούς υπολογιστές. Για παράδειγμα η παραγοντοποίηση σε πρώτους αριθμούς ενός $n-bit$ ακεραίου χρησιμοποιώντας τον καλύτερο δυνατό κλασσικό αλγόριθμο, θα χρειαζόταν $exp({\Theta(n^{1/3}log^{2/3}n)})$. Αυτό επί της ουσίας είναι εκθετικά το μέγεθος του ακεραίου τον οποίο θέλουμε να παραγωγίσουμε. Γι' αυτόν ακριβώς τον λόγο, το πρόβλημα της παραγοντοποίησης θεωρείται άλυτο στους κλασσικούς υπολογιστές. Αντίστοιχα, ένας κβαντικός υπολογιστής, έχει τη δυνατότητα να λύσει το ίδιο πρόβλημα σε $O(n^2 logn log(log n))$, που σημαίνει ότι ένας κβαντικός υπολογιστής μπορεί να λύσει εκθετικά πιο γρήγορα το συγκεκριμένο πρόβλημα. Αυτό μπορεί από μόνο του να είναι εντυπωσιακό, ωστόσο σίγουρα δημιουργούνται τα ερωτήματα του πόσα άλλα προβλήματα μπορεί να λυθούν με τη χρήση κβαντικών υπολογιστών.
Εδώ θα εξετάσουμε τον κβαντικό μετασχηματισμό $Fourier$ ($Quantum Fourier Transformation$), ο οποίος αποτελεί τη βάση για πολλούς κβαντικούς αλγορίθμους.

Στον γρήγορο μετασχηματισμό $Fourier$~\cite{DBLP:books/daglib/0017733} παίρνουμε σαν είσοδο ένα μιγαδικό διάνυσμα Μ-διάστασης, $\alpha$ και σαν έξοδο επιστρέφει ένα μιγαδικό διάνυσμα Μ-διάστασης, $\beta$. Έχουμε δηλαδή το εξής:\\

$\begin{bmatrix}
\beta_0\\
\beta_1\\
\beta_2\\
\vdots\\
\beta_{M-1}
\end{bmatrix}$
=
$\frac{1}{\sqrt{M}}$
$\begin{bmatrix}
1 & 1 & 1 & \dots & 1\\
1 & \omega & \omega^2 & \dots & \omega^{M-1}\\
1 & \omega^2 & \omega^4 & \dots & \omega^{2(M-1)}\\
\ & \ & \vdots & \ & \ \\
1 & \omega^j & \omega^2j & \dots & \omega^{(M-1)j}\\
\ & \ & \vdots & \ & \ \\
1 & \omega^{(M-1)} & \omega{^2(M-1)} & \dots & \omega^{(M-1)(M-1)}\\
\end{bmatrix}$
$\begin{bmatrix}
\alpha_0\\
\alpha_1\\
\alpha_2\\
\vdots\\
\alpha_{M-1}
\end{bmatrix}$\\

,όπου το $\omega$ είναι η Μοστή μιγαδική ρίζα της μονάδας. Οι κλασσικές μέθοδοι θα χρειαζόταν χρόνο $Ο(Μ^2)$, ενώ ο γρήγορος μετασχηματισμός $Fourier (FFT)$, μπορεί να κάνει ακριβώς το ίδιο σε χρόνο $Ο(M logM)$. Παρά την μεγάλη αύξηση στην ταχύτητα που προκαλεί ο $FFT$, ο κβαντικός μετασχηματισμός $Fourier$, $(QFT)$, καταφέρνει να μειώσει και άλλο τον χρόνο εκτέλεσης εκθετικά φτάνοντας τον σε $O(log^2 M)$.

Ο κβαντικός μετασχηματισμός $Fourier$ είναι ο ίδιος μετασχηματισμός με λίγο διαφορετικές συμβάσεις. Επί της ουσίας είναι ένας ορθομοναδιαίος συντελεστής του χώρου $Hilbert$~\cite{RilPerry}. Είναι ένας μετασχηματισμός που αποτελεί τη βάση για πολλούς κβαντικούς αλγορίθμους. Συνοπτικά μετασχηματίζει μια συνάρτηση από το πεδίο ορισμού του χρόνου $(time domain)$, στο πεδίο ορισμού της συχνότητας $(frequency domain)$, δηλαδή συναρτήσεις με περίοδο $r$ σε συναρτήσεις που οι μη μηδενικές τιμές τους είναι μόνο τα πολλαπλάσια της συχνότητας $\frac{1}{r}$~\cite{Calude:2001:CCA:374924}.


Το αμέσως επόμενο ερώτημα που προκύπτει αφορά το πως είναι εφικτό ο $QFT$ να έχει χρόνο μικρότερο από $M$ που είναι το μήκος της εισόδου. Για να είναι εφικτό, κωδικοποιούμε την είσοδο σε μια υπέρθεση μεγέθους $m = log M \ qubits$. Η υπέρθεση αυτή αποτελείται από $2^m$ τιμές πλάτους. Θα μπορούσαμε να γράψουμε την υπέρθεση με τον εξής τρόπο: $\vert \alpha \rangle =  \sum_{j=0}^{M-1} \alpha_j \vert j \rangle$, όπου το $\alpha_i$ είναι το εύρος της δυαδικής συμβολοσειράς  $m-bit$ που αντιστοιχεί στο $i$ με τον φυσικό τρόπο. Αυτό μας οδηγεί σε ένα βασικό σημείο: το $\vert j \rangle$ είναι επί της ουσίας ένας διαφορετικός τρόπος γραφής ενός διανύσματος, όπου ο δείκτης κάθε καταχώρησης γράφεται στο ειδικό σύμβολο της αγκύλης. Ξεκινώντας από την υπέρθεση $\vert \alpha \rangle$, ο $QFT$ τρέχει σε $m = log M$ βήματα. Σε κάθε βήμα, η υπέρθεση εξελίσσεται έτσι ώστε να κωδικοποιεί τα ενδιάμεσα στάδια το ίδιο με τον κλασσικό $FFT$. Αυτό μπορεί να επιτευχθεί με $m$ κβαντικές διεργασίες σε κάθε στάδιο. Επομένως, μετά από $m$ τέτοια στάδια και $m^2 = log^2 M$ βασικές διεργασίες, καταλήγουμε στην υπέρθεση $\vert \beta \rangle$ που ανταποκρίνεται στο επιθυμητό αποτέλεσμα του $QFT$.

Πἐραν αυτού όμως, ο $QFT$ έχει μια βασική διαφορά στο αποτέλεσμα εξόδου του σε σχέση με τον $FFT$. Ο κλασσικός $FFT$, επιστρέφει σαν αποτέλεσμα τους $M$ μιγαδικούς αριθμούς $\beta_0, \beta_1,...,\beta_{M-1}$. Αντίθετα ο $QFT$, επιστρέφει την υπέρθεση $\sum_{j=0}^{M-1} \beta \vert j \rangle$. Τα δεδομένα αυτά όμως δεν είναι προσβάσιμα σε εμάς. Έτσι λοιπόν ο μόνος τρόπος για να αξιοποιήσουμε το αποτέλεσμα, είναι μετρώντας το. Η μέτρηση της κατάστασης του συστήματος αποδίδει μόνο $m = log M$ κλασσικά $bits$. Πιο συγκεκριμένα, η έξοδος είναι ο δείκτης $j$ με πιθανότητα ${\vert \beta_j \vert}^2$. O $QFT$ μπορεί να εφαρμοστεί για αυθαίρετες τιμές του $Μ$ και μπορούμε να τον συνοψίσουμε ως εξής:

\selectlanguage{english}
    \textbf{Input}: A superposition  of $m = log M$ qubits , $ \vert \alpha\rangle = \sum_{j=0}^{M-1} \alpha_j \vert j \rangle$\\
    \textbf{Method}: Using $O(m^2) = O(log^2 M)$ quantum operations perform the quantum FFT to obtain the superposition  $\vert \beta \rangle = \sum{j=0}^{M-1} \beta_j \vert j \rangle$.\\
    \textbf{Output}: A random m-bit number j, from the probability distribution $Pr[j] = {\vert \beta_j \vert}^2$.

\selectlanguage{greek}
Ο κβαντικός μεταασχηματισμός $Fourier$ ορίζεται σε μια ορθοκανονική βάση $\vert 0 \rangle,...,  \vert N \rangle$ ως εξής~\cite{Nielsen:2011:QCQ:1972505}:

\begin{align*}
  \vert j \rangle \rightarrow \frac{1}{\sqrt{N}} \sum_{j=0}^{N-1} e^{\frac{2 \pi ijk}{N}}\vert k \rangle
\end{align*}

και άρα γράφεται:

\begin{align*}
  \sum_{j=0}^{N-1}x_j \vert j \rangle \rightarrow \sum_{k=0}^{N-1} y_k \vert k \rangle
\end{align*}
όπου τα πλάτη $y_k$ είναι τα αντίστοιχα πλάτη $x_j$ του διακριτού μετασχηματισμού $Fourier$. Ο μετασχηματισμός αυτός είναι $unitary$ και έτσι μπορεί να αποτελέσει τη βάση για έναν κβαντικό υπολογιστή. Μπορεί να αποδειχθεί ως εξής:
Έστω $N = 2^n$, όπου $n$ κάποιος ακέραιος και η βάση $\vert 0 \rangle, ..., \vert 2^n \rangle$ είναι η υπολογιστική βάση για έναν κβαντικό υπολογιστή $n qubit$. Για την κατάσταση $\vert j \rangle$, χρησιμοποιούμε τη δυαδική αναπαράσταση $j = j_1j_2...j_n = j_12^{n-1} + j_2n^{n-2} + ... + j_n2^0$. Επίσης η έκφραση $0._{j_1 j_{l=1}...j_n}$ θα αναπαριστά το δυαδικό κλάσμα $\frac{j_1}{2} + \frac{j_{l+1}}{4} + ... + \frac{j_m}{2^{m-l+1}}$.
Από τη γραμμική άλγεβρα, μπορούμε να γράψουμε τον κβαντικό μετασχηματισμό $Fourier$ με τον εξής τρόπo:

\begin{align*}
  \vert j_1, ... ,j_n \rangle \rightarrow \frac{(\vert 0 \rangle + e^{2 \pi i0._{j_1}} \vert 1 \rangle)(\vert 0 \rangle + e^{2 \pi i0._{j_1 j_n}} \vert 1 \rangle)...(\vert 0 \rangle + e^{2 \pi i0._{j_1 j_2 ... j_n}} \vert 1 \rangle)}{2^{\frac{n}{2}}}
\end{align*}

Η παραπάνω αναπαράσταση είναι τόσο χρήσιμη που θα μπορούσε να χρησιμοποιηθεί και σαν τον ορισμό του $QFT$.
Οι παραπάνω δύο σχέσεις μπορούν να γραφούν και έτσι:

\begin{flalign}
  \vert j \rangle &\rightarrow \frac{1}{2^{\frac{n}{2}}} \sum_{k=0}^{2^n-1} e^{\frac{2 \pi ijk}{2^n}} \vert k \rangle &&\\
  &= \frac{1}{2^{\frac{n}{2}}} \sum_{k_1=0}^{1}...\sum_{k_n=0}^{1}e^{2 \pi ij(\sum_{l=1}^{n}k_l2^{-l})}\vert k_1...k_n \rangle&&\\
  &= \frac{1}{2^{\frac{n}{2}}} \sum_{k_1=0}^{1}...\sum_{k_n=0}^{1} \bigotimes e^{2 \pi ijk_l2^{-l}}\vert k_l \rangle&&\\
  &= \frac{1}{2^{\frac{n}{2}}} \bigotimes [\sum_{k_n=0}^{1} e^{2 \pi ijk_l2^{-l}}\vert k_l \rangle]&&\\
  &= \frac{1}{2^{\frac{n}{2}}} \bigotimes [\vert 0 \rangle e^{2 \pi ij2^{-l}}\vert 1 \rangle]&&\\
  &= \frac{(\vert 0 \rangle + e^{2 \pi i0._{j_n} \vert 1 \rangle})(\vert 0 \rangle + e^{2 \pi i0._{j_{n-1}j_n} \vert 1 \rangle})(\vert 0 \rangle + e^{2 \pi i0._{j_1 j_2...j_n} \vert 1 \rangle})}{2^{\frac{n}{2}}}&&
\end{flalign}

Και βλέπουμε ότι όντως ο $QFT$ είναι $unitary$.

Εκτός από τις βασικές καταστάσεις, ο κβαντικός μετασχηματισμός $Fourier$ μπορεί να δράσει και σε υπερθέσεις των βασικών καταστάσεων ενός κβαντικού καταχωρητή.
Έστω η παρακάτω υπέρθεση βασικών καταστάσεων:

\begin{align*}
  x_0\vert 0 \rangle + x_1\vert 1 \rangle + ... + x_a\vert a \rangle + ... + x_{N-1} \vert N-1 \rangle = \sum_{a=0}^{N-1} x_a \vert a \rangle
\end{align*}

Ο κβαντικός μετασχηματισμός της υπέρθεσης, δίνεται από το παρακάτω:

\begin{align*}
  \sum_{a=0}^{N-1}x_a \vert a \rangle \mapsto \frac{1}{\sqrt{N}}\sum_{c=0}^{N-1} \sum_{a=0}^{N-1} x_a e^{2 \pi i \frac{ac}{N}} \vert c \rangle = \frac{1}{\sqrt{N}} \sum_{c=0}^{N-1} y_c \vert c \rangle
\end{align*}

όπου το $y_c$ είναι ο κλασσικός μετασχηματισμός $Fourier$ του $x_a$ που δίνεται από:

\begin{align*}
  y_c = \sum_{a=0}^{N-1} x_a e^{a \pi i\frac{ac}{N}}
\end{align*}

Για να γίνει ακόμα πιο κατανοητός ο $QFT$, μπορούμε να δούμε τα τρία παρακάτω παραδείγματα.

\begin{example}
Η επίδραση του $QFT$ σε ένα $qubit$.\\
Αρχικά υποθέτουμε ότι το $qubit$ βρίσκεται στην κατάσταση $\vert 0 \rangle$.
Άρα σύμφωνα με τα παραπάνω:
\begin{align*}
  \vert 0 \rangle \mapsto \frac{1}{\sqrt{2}} \sum_{c=0}^{1} e^{2 \pi i \frac{0c}{2}} \vert c \rangle = \frac{1}{\sqrt{2}}(\vert 0 \rangle + \vert 1 \rangle)
\end{align*}
Αν όμως το $qubit$ βρίσκεται στην κατάσταση $\vert 1 \rangle$, τότε:
\begin{align}
  \vert 1 \rangle \mapsto \frac{1}{\sqrt{2}} \sum_{2 \pi i \frac{1c}{2}} \vert c \rangle = \frac{1}{\sqrt{2}}(\vert 0 \rangle + e^{\pi i} \vert 1 \rangle) = \frac{1}{\sqrt{2}}(\vert 0 \rangle - \vert 1 \rangle)
\end{align}
 Όπως βλέπουμε, ο κβαντικός μετασχηματισμός $Fourier$ δρα σε ένα $bit$ όπως η πύλη $H$. Για τον λόγο αυτό, η πύλη $H$ αναφέρεται και ως "μετασχηματισμός $Hadamard$".
\end{example}

\begin{example}
  Εδώ θα δούμε τον $QFT$ σε έναν κβαντικό καταχωρητή που αποτελείται από δύο $qubits$ και βρίσκεται στην κατάσταση $\vert 10 \rangle$ που αντιστοιχεί στο δεκαδικό $\vert 2 \rangle$.
  \begin{align}
    \vert 2 \rangle \mapsto \frac{1}{\sqrt{4}}\sum_{c=0}^{3}e^{2 \pi i \frac{2c}{4}}\vert c \rangle = \frac{1}{2}(\vert 0 \rangle + e^{\pi i}\vert 1 \rangle + e^{3 \pi i}\vert 2 \rangle + e^{3 \pi i} \vert 3 \rangle) = \frac{1}{2}(\vert 0 \rangle - \vert 1 \rangle - \vert 3 \rangle)
  \end{align}
  Αν αντίστοιχα ο κβαντικός καταχωρητής βρίσκεται στην κατάσταση $\vert 01 \rangle$, τότε:
 \begin{align}
  \vert 1 \rangle \mapsto \frac{1}{\sqrt{4}}\sum_{c=0}^{3}e^{2 \pi i \frac{2c}{4}}\vert c \rangle = \frac{1}{2}(\vert 0 \rangle + e^{\frac{\pi i}{2}}\vert 1 \rangle + e^{ \pi i}\vert 2 \rangle + e^{\frac{3 \pi i}{2}} \vert 3 \rangle) = \frac{1}{2}(\vert 0 \rangle - i\vert 1 \rangle - \vert 2 \rangle _ i\vert 3 \rangle)
 \end{align}

\end{example}

\begin{example}
 Εδώ το ίδιο παράδειγμα σε έναν καταχωρητή με τρία $qubits$ που βρίσκεται στην κατάσταση $\vert 010 \rangle$ δηλαδή $\vert 2 \rangle$, στο δεκαδικό.
 \begin{flalign}
    \vert 2 \rangle &\mapsto \frac{1}{\sqrt{8}}\sum_{c=0}^{7}e^{2 \pi i \frac{2c}{8}}\vert c \rangle&&\\
    &= \frac{1}{\sqrt{8}}(\vert 0 \rangle + e^{\frac{\pi i}{2}}\vert 1 \rangle + e^{\pi i}\vert 2 \rangle + e^{\frac{3 \pi i}{2}} \vert 3 \rangle + e^{2 \pi i} \vert 4 \rangle + e^{\frac{5 \pi i}{2}} \vert 5 \rangle + e^{3 \pi i}\vert 6 \rangle + e^{\frac{7 \pi i}{2}} \vert 7 \rangle) &&\\
     &= \frac{1}{\sqrt{8}}(\vert 0 \rangle + i\vert 1 \rangle - \vert 2 \rangle - i\vert 3 \rangle + \vert 4 \rangle + i\vert 5 \rangle - \vert 6 \rangle -i\vert 7 \rangle)
 \end{flalign}
\end{example}

Από όλα τα παραπάνω μπορούμε να δούμε ότι ο κβαντικός μετασχηματισμός $Fourier$ μετασχηματίζει μια βασική κατάσταση ενός κβαντικού καταχωρητή σε υπέρθεση όλων των βασικών καταστάσεων, όπου όλες οι βασικές καταστάσεις έχουν το ίδιο πλάτος πιθανότητας αλλά διαφορετικές φάσεις.

Ο $QFT$ μας βοηθάει να βρίσκουμε με ευκολία την περίοδο μιας περιοδικής συνάρτησης. Σε έναν κβαντικό καταχωρητή μεγέθους $q bits$, που βρίσκεται σε υπέρθεση περιοδικών καταστάσεων	περιόδου $r$, με τον $QFT$ στον κβαντικό καταχωρητή θα μείνουν μόνο οι τιμές που είναι πολλαπλάσια του $\frac{q}{r}$. Αυτά τα πολλαπλάσια είναι ισοπίθανα άρα μια μέτρηση θα μας επιστρέψει ένα από αυτά.~\cite{DBLP:books/daglib/0017733}.

Προσπαθώντας να συνοψίσουμε κάπως όλα τα παραπάνω, ο κλασσικός γρήγορος μετασχηματισμός $Fourier (FFT)$ έχει πλυποκότητα $O(NlogN) = O(2^n n)$ με $N = 2^n$, είναι δηλαδή πολύ γρήγορος. Ωστόσο ο κβαντικός μετασχηματισμός $Fourier$ είναι ακόμα πιο γρήγορας με πολυπλοκότητα $O(log^2 N) - O(n^2)$~\cite{DawarAnuj}.

Η βασική τους διαφορά είναι ότι ο $FFT$ επιστρέφει τη σωστή απάντηση, ενώ ο $QFT$ υπολογίζει την υπέρθεση τως σωστής απάντησης.

\begin{align}
  \vert q \rangle = \sum_{i=0}^{2^n -1} c_i \vert c \rangle
\end{align}
 με την μέτρηση του παραπάνω προκύπτει ένας αριθμός $n-bit$ με πιθανότητα $\vert c_i \vert^2$

\subsection{Περιοδικότητα}
\selectlanguage{greek}
Ο $QFT$, θα μπορούσαμε να πούμε ότι είναι ένας γρήγορος τρόπος για να πάρει κάποιος μια γενική ιδέα του $FFT$. Ανιχνέυουμε ένα από τα μεγαλύτερα στοιχεία του διανλύσματος της εξόδου, χωρίς όμως να μπορούμε να δούμε τίποτα για αυτό πέραν του δείκτη του.

Έστω ότι η είσοδος του $QFT$, $\vert \alpha \rangle = (\alpha_0,...,\alpha_{M-1})$, τέτοια ώστε $\alpha_i = \alpha_j$ κάθε φορά που $i \equiv j mod k$, όπου $k$ είναι ένας ακέραιος ο οποίος διαιρεί το $Μ$. Δηλαδή ο πίνακας $\alpha$ αποτελείται από $M/k$ επαναλήψεις κάποιας ακολουθίας $(\alpha_0,...,\alpha_{k-1})$, μήκους $k$. Ας υποθέσουμε ότι μόνο ένας από τους $k$ αριθμούς είναι μη μηδενικός, ας πούμε ο $\alpha_j$. Τότε λέμε ότι το $\vert \alpha \rangle$, είναι περιοδικό, με περίοδο $k$ και μετατόπιση $j$. Αυτό σημαίνει ότι εάν το διάνυσμα εισόδου είναι περιοδικό, τότε μπορούμε να χρησιμοποιήσουμε τον κλασσικό $FFT$ για να υπολογίσουμε την περίοδο του. Προκύπτει λοιπόν ο εξής ορισμός:\\

\textit{
Υποθέτουμε ότι η είσοδος στον $QFT$ είναι περιοδική με περίοδο $k$, για κάποια $k$ που διαιρούν το $M$. Τότε η έξοδος θα είναι πολλαπλάσιο του $\frac{M}{k}$ και είναι δυνατό να είναι οποιοδήποτε από τα $k$ πολλαπλάσια του $\frac{M}{k}$.}\\


Αυτό που μπροούμε να δούμε εδώ, είναι πως με την πολλαπλή επανάληψη της δειγματοληψίας και στη συνέχεια επιλέγοντας τον μεγαλύτερο κοινό διαιρέτη όλων των δεικτών που επιστράφηκαν, έχουμε πολύ μεγάλη πιθανότητα να πάρουμε τον $\frac{M}{k}$ και έτσι να είναι εφικτό να βρούμε την περίοδο $k$ της εισόδου.

\selectlanguage{english}
\begin{remark}
\selectlanguage{greek}
Έστω ότι έχουμε $s$ ανεξάρτητα δείγματα τα οποία έχουν σχεδιαστεί ομοιόμορφα από
\begin{align*}
    0, \frac{M}{k}, \frac{2M}{k},...,\frac{(k-1)M}{k}
\end{align*}
Τότε με πιθανότητα τουλάχιστον $1-\frac{k}{2^s}$, ο μέγιστος κοινός διαιρέτης όλως αυτών των δειγμάτων είναι το $\frac{M}{k}$.
\end{remark}

\selectlanguage{greek}
Ο μόνος τρόπος για να μην συμβεί αυτό, είναι όλα τα δείγματα να είναι πολλαπλάσια του $j \cdot \frac{M}{k}$, όπου το $j$ είναι ακέραιος μεγαλύτερος του 1. Η πιθανότητα ένα τυχαίο δείγμα να είναι πολλαπλάσιο του $j \cdot \frac{M}{k}$ είναι το πολύ $\frac{1}{j} \leq \frac{1}{2}$ και η πιθανότητα να είναι όλα τα δείγματα πολλαπλάσια του $j \cdot \frac{M}{k}$ είναι το μέγιστο $\frac{1}{2^s}$. Όλα αυτά ισχύουν για συγκεκριμένο αριθμό $j$. Η πιθανότητα αυτό να συμβεί για μερικά $j$, όπου $j \leq k$ είναι το πολύ ίση με το άθροισμα όλων αυτών των πιθανοτήτων πάνω από τις διαφορετικές τιμές του $j$, που δεν είναι περισσότερες από $\frac{k}{2^s}$. Μπορούμε να μειώσουμε την πιθανότητα αποτυχίας επιλέγοντας το $s$ έτσι ώστε να είναι κατάλληλο πολλαπλάσιο του $log M$.
% Όπως είδαμε αναλυτικά παραπάνω, ο διακριτός μετασχηματισμός $Fourier$ έχει σαν είσοδο ένα διάνυσμα μιγαδικών αριθμών ($x_0, ..., x_N-1$), όπου $N$ είναι το μήκος τους διανύσματος. Η έξοδος μετά τη μετατροπή που εφαρμόζει ο αλγόριθμος είναι ένα διάνυσμα $y_0, ..., y_N-1$, το οποίο ορίζεται ως εξής:
%
% Ο κβαντικός μετασχηματισμός $Fourier$,
