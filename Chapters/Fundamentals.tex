\chapter{\selectlanguage{english}Fundamentals}

\section{\tl{Overview}}

\section{\tl{Quantum Computation}}

\subsection{\tl{Qubits}}

\subsection{\tl{Quantum Circuits}}

\section{\tl{The Dirac notation nad Hilbert spaces}}

\section{\tl{Quantum Circuits}}

\subsection{\tl{The quantum Circuit model}}

\subsection{\tl{Quantum Gates}}
\selectlanguage{greek}
Όπως είναι γνωστό οι πύλες των κλασσικών υπολογιστών είναι φυσικά συστήματα. Αντίθετα στους κβαντικούς υπολογιστές, όταν μιλάμε για κβαντικές πύλες, συνήθως δεν αναφερόμαστε σε φυσικά συστήματα. Οι κβαντικές πύλες αντιπροσωπέυουν δράσεις που ασκούνται σε $qubits$ ή σε κβαντικούς καταχωρητές, οι οποίες αντιπροσωπεύονται από τελεστές του χώρου
$Hilbert$\cite{Kaye:2007:IQC:1206629}. Σε αντίθεση με τους κλασσικούς υπολογιστές όπου η πληροφορία διέρχεται μέσα από τις πύλες, στους κβαντικούς υπολογιστές η πληροφορία δεν διέρχεται μέσα από τις κβαντικές πύλες. Είναι αποθηκευμένη σε $qubits$ ή κβαντικούς καταχωρητές και παραμένει εκεί. Οι κβαντικές πύλες δρουν πάνω σε αυτά τα $qubits$ αλλάζοντας την κατάσταση τους.
\\
Θα μπορούσαμε να χωρίσουμε τις κβαντικές πύλες σε τρεις κατηγορίες ανάλογα με τον αριθμό των $qubits$ στα οποία επιδρούν. Έχουμε δύο βασικές κατηγορίες λοιπόν:
\begin{enumerate}
  \item Πύλες που επιδρούν σε ένα $qubit$
  \item Πύλες που επιδρούν σε δυο $qubits$
  \item Πύλες που επιδρούν σε τρια $qubits$
\end{enumerate}

\subsubsection{Κβαντικές Πύλες που δρουν σε ένα \tl{qubit}}
\subsubsection{Κβαντική πύλη αδράνειας}
Η κβαντική πύλη αδράνειας συμβολίζεται με $I$ και περιγράφεται από έναν τελεστή που ονομάζεται τελεστής αδράνειας. Ο πίνακας της πύλης είναι ο παρακάτω:
\\
\begin{flalign*}
  I &= \begin{bmatrix}
    1 & 0\\
    0 & 1
\end{bmatrix}
\end{flalign*}
Η κβαντική πύλη αδράνειας, αφήνει αμετάβλητη την κατάσταση του $qubit$. Οι επιδράσεις της πύλης στις καταστάσεις ενός $qubit$ φαίνονται στον πίνακα. Στην πρώτη στήλη φαίνονται οι καταστάσεις του $qubits$ πριν την επίδραση της πύλης και στη δεύτερη στήλη φαίνονται οι καταστάσεις τους μετά την επίδραση της πύλης.


\begin{tabular}{lllll}
\rowcolor[HTML]{C0C0C0}
 &$\vert q_1 \rangle$  & $\vert q_0 \rangle$ \\
 &$\vert 0 \rangle$  & $\vert 0 \rangle$ \\
 &$\vert 1 \rangle$  & $\vert 1 \rangle$ \\
 &$\vert q \rangle$  & $\vert q \rangle$
\end{tabular}

\subsubsection{Πύλη Μετατόπισης}
Ο τελεστής $\Phi$ αντιστοιχεί στον παρακάτω πίνακα:
\\
\begin{flalign*}
  \Phi &= \begin{bmatrix}
    1 & 0\\
    0 & e^{i \varphi}
\end{bmatrix}
\end{flalign*}

Έστω ένα $qubit$ $\vert q_1 \rangle$, η κατάσταση του οποίου είναι:
\\
\begin{align*}
  \vert q_1 \rangle = \alpha \vert 0 \rangle + b \vert 1 \rangle &= \begin{bmatrix}
    \alpha \\
    b
\end{bmatrix}
\end{align*}

Μετά την επίδραση της κβαντικής πύλης μετατόπισης, η κατάσταση αλλάζει σε $\vert q_0 \rangle$:
\begin{align*}
  \vert q_0 \rangle = \Phi \vert q_1 \rangle
  &= \begin{bmatrix}
  1 & 0 \\
  0 & e^{i \varphi}
\end{bmatrix}
\begin{bmatrix}
  \alpha \\
  b
\end{bmatrix}
= \begin{bmatrix}
  \alpha\\
  e^{i \varphi}b
\end{bmatrix}
\end{align*}
Άρα η νέα κατάσταση του $qubit$ είναι $\vert q_0 \rangle = \alpha \vert 0 \rangle + e^{i \varphi} b \vert 1 \rangle$.
Και ο πίνακας είναι ο εξής:
\begin{tabular}{lllll}
\rowcolor[HTML]{C0C0C0}
 &$\vert q_1 \rangle$  & $\vert q_0 \rangle$ \\
 &$\vert 0 \rangle$  & $\vert 0 \rangle$ \\
 &$\vert 1 \rangle$  & $e^{i \varphi}\vert 1 \rangle$ \\
 &$\alpha \vert 0 \rangle + b \vert 1 \rangle$  & $\alpha \vert 0 \rangle + e^{i \varphi} b \vert 1 \rangle$
\end{tabular}

\subsubsection{Πύλη \tl{Hadamard}}
Η κβαντική πύλη $Hadamard$ είναι ίσως η πιο γνωστή κβαντική πύλη, καθώς χρησιμοποιείται σε πολλούς αλγορίθμους.
\begin{flalign*}
  H &= \frac{1}{\sqrt{2}}
  \begin{bmatrix}
    1 & 1\\
    1 & -1
  \end{bmatrix}
\end{flalign*}

Η δράση της πύλης είναι η εξής:
\begin{align*}
  H \vert 0 \rangle &=
  \begin{bmatrix}
    \frac{1}{\sqrt{2}} & \frac{1}{\sqrt{2}}\\
    \frac{1}{\sqrt{2}} & -\frac{1}{\sqrt{2}}
  \end{bmatrix}
  \begin{bmatrix}
    1 \\
    0
  \end{bmatrix}
  = \begin{bmatrix}
    \frac{1}{\sqrt{2}}\\
    \frac{1}{\sqrt{2}}
  \end{bmatrix}
  = \frac{1}{\sqrt{2}} \vert 0 \rangle + \frac{1}{\sqrt{2}} \vert 1 \rangle = \frac{1}{\sqrt{2}}(\vert 0 \rangle + \vert 1 \rangle)
\end{align*}

Αυτό είναι το αποτέλεσμα σε ένα $qubit$ με αρχική κατάσταση 0.
Σε ένα $qubit$ με αρχική κατάσταση 1, η τελική κατάσταση θα είναι
\begin{align*}
  \frac{1}{\sqrt{2}}(\vert 0 \rangle - \vert 1 \rangle)
\end{align*}
Επομένως η πύλη $Hadamard$, θέτει το οποιοδήποτε $qubit$ πάνω στο οποίο επιδρά, σε υπέρθεση των δύο βασικών καταστάσεων. Η πιθανότητα να το μετρήσουμε και να το βρούμε σε μια από τις δυο βασικές καταστάσεις ισούται με 0.5.

Έαν όμως ένα $qubit$ βρίσκεται σε μια υπέρθεση καταστάσεων, τότε έχουμε το εξής:

\begin{align*}
  H \frac{1}{\sqrt{2}}(\vert 0 \rangle + \vert 1 \rangle)
  &= \begin{bmatrix}
    \frac{1}{\sqrt{2}} & \frac{1}{\sqrt{2}}\\
    \frac{1}{\sqrt{2}} & -\frac{1}{\sqrt{2}}
  \end{bmatrix}
  \begin{bmatrix}
    \frac{1}{\sqrt{2}}\\
    \frac{1}{\sqrt{2}}
  \end{bmatrix} = \begin{bmatrix}
      1\\
      0
  \end{bmatrix} = \vert 0 \rangle
\end{align*}

Και εάν το  $qubit$ είναι στην υπέρθεση του δεύτερου αποτελέσματος, η επίδραση θα δίνει επίσης το αντίθετο αποτέλεσμα, που ισούται με $\vert 1 \rangle$.

Συνοπτικά λοιπόν τα αποτελέσματα
\\
\begin{tabular}{lllll}
\rowcolor[HTML]{C0C0C0}
 &$\vert q_1 \rangle          $  & $\vert q_0 \rangle$ \\
 &$\vert 0 \rangle            $  & $\frac{1}{\sqrt{2}}(\vert 0 \rangle + \vert 1 \rangle)$ \\
 &$\vert 1 \rangle$  & $\frac{1}{\sqrt{2}}(\vert 0 \rangle - \vert 1 \rangle)$ \\
 &$\frac{1}{\sqrt{2}}(\vert 0 \rangle + \vert 1 \rangle)$  & $\vert 0 \rangle$\\
 &$\frac{1}{\sqrt{2}}(\vert 0 \rangle - \vert 1 \rangle)$  & $\vert 1 \rangle$
\end{tabular}

\subsubsection{Πύλες που δρουν σε δυο \tl{qubits}}

\subsubsection{Η πύλη ελεγχόμενου ΟΧΙ}

Η συγκεκριμένη πύλη επιδρά σε 2 $qubits$ και ο τελεστής της συμβολίζεται με $CNOT$. Από τα δύο $qubit$ στα οποία επιδρά, το ένα συμβολίζεται με $c$ και ονομάζεται $qubit$ ελέγχου και το δεύτερο με $t$ και ονομάζεται $qubit$ στόχος.Η πύλη $CNOT$, αλλάζει την κατάσταση του $qubit$  στόχου όταν το $qubit$ ελέγχου είναι $\vert 1 \rangle$ και την αφήνει ίδια στην αντίστροφη περίπτωση.
Η γενική περιγραφή της πύλης είναι:

\begin{align*}
  CNOT \vert c_i t_i \rangle = \vert c_o t_o \rangle
\end{align*}

Έστω ότι η κατάσταση του κβανικού καταχωρητή πριν τη δράση της πύλης είναι $\vert 1 0 \rangle$. Μετά την επίδραση της πύλης θα γίνει $\vert 11 \rangle$.

Φτιάχνοντας τον πίνακα των αποτελεσμάτων:
\\
\begin{tabular}{lllll}
\rowcolor[HTML]{C0C0C0}
 &$\vert c_i t_i \rangle          $  & $\vert c_0 t_0 \rangle$ \\
 &$\vert 00 \rangle            $  & $\vert 00 \rangle $ \\
 &$\vert 01 \rangle$  & $\vert 01 \rangle$ \\
 &$\vert 10 \rangle$  & $\vert 11 \rangle$\\
 &$\vert 11 \rangle$  & $\vert 10 \rangle$
\end{tabular}

Ένα σημαντικό στοιχείο, είναι ότι οι τρείς πύλες που έχουμε δεί, η $CNOT$, η $Η$ και η $\Phi$ αποτελούν ένα γενικευμένο σύνολο κβαντικών πυλών. Αυτό σημαίνει ότι με τις πύλες αυτές και μόνο αυτές, μπορούμε να εκτελέσουμε οποιοδήποτε κβαντικό υπολογισμό.


\subsubsection{Πύλη ελεγχόμενης μετατόπισης φάσης}

Οι πιο συνηθισμένοι τελεστές αναφοράς της συγκεκριμένης πύλης είναι οι $S, CP, C \Phi$. Όπως και στην προηγούμενη πύλη, υπάρχει το $qubit$ ελέγχου και το $qubit$ στόχος. Η πύλη $C \Phi$, πολλαπλασιάζει την κατάσταση του $qubit$  στόχου με τον παράγοντα φάσης $e^{i \varphi}$ μόνο όταν η κατάσταση του $qubit$ ελέγχου και η κατάσταση του $qubit$ στόχου είναι $\vert 1 \rangle$.
Η γενική περιγαφή της πύλης είναι
\begin{align*}
  C \Phi \vert c_i t_i \rangle = \vert c_o t_o \rangle
\end{align*}

Φτιάχνοντας τον πίνακα των αποτελεσμάτων:
\\
\begin{tabular}{lllll}
\rowcolor[HTML]{C0C0C0}
 &$\vert c_i t_i \rangle          $  & $\vert c_0 t_0 \rangle$ \\
 &$\vert 00 \rangle            $  & $\vert 00 \rangle $ \\
 &$\vert 01 \rangle$  & $\vert 01 \rangle$ \\
 &$\vert 10 \rangle$  & $\vert 10 \rangle$\\
 &$\vert 11 \rangle$  & $e^{i \varphi} \vert 11 \rangle$
\end{tabular}
