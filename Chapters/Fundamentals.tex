\chapter{\selectlanguage{english}Fundamentals}

\section{\tl{Overview}}

\section{\tl{Quantum Computation}}

\subsection{\tl{Qubits}}

\subsection{\tl{Quantum Circuits}}

\section{\tl{The Dirac notation nad Hilbert spaces}}

\section{\tl{Quantum Circuits}}

\subsection{\tl{The quantum Circuit model}}

\subsection{\tl{Quantum Gates}}
\selectlanguage{greek}
Όπως είναι γνωστό οι πύλες των κλασσικών υπολογιστών είναι φυσικά συστήματα. Αντίθετα στους κβαντικούς υπολογιστές, όταν μιλάμε για κβαντικές πύλες, συνήθως δεν αναφερόμαστε σε φυσικά συστήματα. Οι κβαντικές πύλες αντιπροσωπέυουν δράσεις που ασκούνται σε $qubits$ ή σε κβαντικούς καταχωρητές, οι οποίες αντιπροσωπεύονται από τελεστές του χώρου
$Hilbert$\cite{Kaye:2007:IQC:1206629}. Σε αντίθεση με τους κλασσικούς υπολογιστές όπου η πληροφορία διέρχεται μέσα από τις πύλες, στους κβαντικούς υπολογιστές η πληροφορία δεν διέρχεται μέσα από τις κβαντικές πύλες. Είναι αποθηκευμένη σε $qubits$ ή κβαντικούς καταχωρητές και παραμένει εκεί. Οι κβαντικές πύλες δρουν πάνω σε αυτά τα $qubits$ αλλάζοντας την κατάσταση τους.
\\
Θα μπορούσαμε να χωρίσουμε τις κβαντικές πύλες σε τρεις κατηγορίες ανάλογα με τον αριθμό των $qubits$ στα οποία επιδρούν. Έχουμε τρεις κατηγορίες λοιπόν:
\begin{enumerate}
  \item Πύλες που επιδρούν σε ένα $qubit$
  \item Πύλες που επιδρούν σε δυο $qubits$
  \item Πύλες που επιδρούν σε τρια $qubits$
\end{enumerate}

\subsection{Κβαντικές Πύλες που δρουν σε ένα \tl{qubit}}
\subsubsection{Κβαντική πύλη αδράνειας}
Η κβαντική πύλη αδράνειας συμβολίζεται με $I$ και περιγράφεται από έναν τελεστή που ονομάζεται τελεστής αδράνειας. Ο πίνακας της πύλης είναι ο παρακάτω:
\begin{flalign*}
  I &= \begin{bmatrix}
    1 & 0\\
    0 & 1
\end{bmatrix}
\end{flalign*}
Η κβαντική πύλη αδράνειας, αφήνει αμετάβλητη την κατάσταση του $qubit$. Οι επιδράσεις της πύλης στις καταστάσεις ενός $qubit$ φαίνονται στον πίνακα:

\begin{table}[]
\begin{tabular}{lllll}
\rowcolor[HTML]{C0C0C0}
 &  &  &  &  \\
 &  &  &  &  \\
 &  &  &  &  \\
 &  &  &  &
\end{tabular}
\end{table}

%\arrayrulecolor[HTML]{DB5800}

% \begin{tabular}{ |l|p{3cm}|p{3cm}|  }
% \hline
% \rowcolor{lightgray} \multicolumn{3}{|c|}{Country List} \\
% \hline
% Country Name    or Area Name& ISO ALPHA 2 Code &ISO ALPHA 3 \\
% \hline
% Afghanistan & AF &AFG \\
% \rowcolor{gray}
% Aland Islands & AX & ALA \\
% Albania   &AL & ALB \\
% Algeria  &DZ & DZA \\
% American Samoa & AS & ASM \\
% Andorra & AD & \cellcolor[HTML]{AA0044} AND    \\
% Angola & AO & AGO \\
% \hline
% \end{tabular}


\subsection{\tl{Universal Quantum Gates}}
